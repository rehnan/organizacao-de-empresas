%Exemplo de capitulo

\chapter{Introdução}

A aquisição de produtos pela internet, está se tornando cada vez mais comum não só no Brasil mas no mundo todo. Segundo dados da E-bit, 51,3 milhões de pessoas já utilizaram a internet ao menos uma vez para adquirir produtos \cite{Campi2014}. Segundo a Associação Brasileira de Comércio Eletrônico, As expectativas para o setor são positivas, pelo fato de que os consumidores brasileiros estão mais confiantes e mais seguros para efetuar compras \textit{online}, pois as novas gerações nasceram praticamente em meio a essa tecnologia, o que facilita e contribui para o crescimento do comércio eletrônico. O crescimento contínuo do comércio eletrônico está sendo impulsionado principalmente por: Maior utilização da Internet; Aumento do hábito de compras online; crescimento da banda larga e sortimento em lojas tradicionais. No Brasil, o comércio eletrônico tem quinze anos de vida \nocite{DIGiorgi2013}. As vendas pela internet no Brasil estão em alta. Para 2015, a E-bit prevê que o e-commerce termine o ano com um faturamento de R\$ 43 bilhões, 20\% maior que o ano anterior \cite{E-Bit2015}.
