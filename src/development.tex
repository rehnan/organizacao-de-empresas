\chapter{Desenvolvimento}

Neste capítulo são descritos algumas fatores essenciais e seus devidos custos para se iniciar um comércio eletrônico.

\subsection{Conceito de E-Commerce}

[Precisa terminar]

\section{Estrutura básica de um Comércio Eletrônico}

Antes de expandir o negócio físico para o universo \textit{online}, é importante analisar alguns fatores essenciais, como: a definição da plataforma de e-commerce, escolha do provedor de hospedagem, pagamento digital, logística (caso seja necessário) e custos operacionais.  Estes fatores serão melhores descritos nas próximas seções \cite{Roitman2010}.

\subsection{Custos Plataforma de E-commerce}

No varejo eletrônico, a plataforma de e-commerce é o sistema responsável pelo gerenciamento e visualização da loja na \textit{Web}. É por meio da plataforma que ocorrerá todo o processo de venda, desde a escolha do produto, disponibilização de soluções de pagamento, efetuação da comprar, geração de notas fiscais, controle de estoque, acompanhamento logístico, entre outros processos mais. Uma plataforma de e-commerce completa, além de possuir ferramentas básicas para a gestão da loja virtual, também deve possuir ferramentas que possibilitem a integração de outras tarefas também importantes para o sucesso de uma loja virtual como \cite{Valle2014}.

Antes da escolha da plataforma, é extremamente importante realizar uma análise sobre os valores que se está disposto a investir na plataforma de e-commerce. Essa decisão deve ser considerada no planejamento estratégico. Existem diversas plataformas de código aberto e privadas. Segundo um artigo sobre as 7 melhores plataformas de 2015, publicado por Tom Hardy, aponta que a plataforma Magento é a líder global, com a quase 26,3\% do mercado, seguido por WooCommerce, com 20\%, Prestashop (7,4\%), Shopify (4,9\%), osCommerce (3,8\%), BigCommerce (3,2\%) e OpenCart (3,1\%) \cite{Hardy2015}.

As plataformas de e-commerce podem ser categorizadas basicamente em 3 tipos:

\begin{enumerate}
	\item Plataformas open source;
	\item Plataformas alugadas – SaaS;
	\item Plataformas exclusivas.
\end{enumerate}


\textbf{1. Plataformas \textit{open sources}} -  Nas plataformas e-commerces de código aberto, o sistemas é código livre, ou seja, o ambiente está disponível para ser baixado e instalado, existem inúmeras opções disponíveis no mercado, dentre elas o Magento, citado anteriormente. Porém, é necessário se ter um bom conhecimento de programação para mexer com o ambiente e poder utilizar os recursos disponíveis por esta plataforma. Nesse caso, será necessário contratar um programador ou empresa especializada. Utilizando a solução Magento Commerce, um profissional bem experiente ou um empresa especializada cobra entre à R\$ 5.000 à R\$ 6.000 para instalar e configurar o sistema.

\textbf{2. Plataformas alugadas – SaaS} - A categoria de lojas virtuais alugadas, a dispersão de preços é muito grande. São diversos fornecedores com várias propostas distintas, e por isso é preciso analisar detalhadamente cada uma para não errar na escolha. Diversos fatores devem ser levados em consideração. Utilizando este tipo de solução (Plataformas alugadas – SaaS,) podemos adiantar que o inicial começa por volta de R\$ 30,00 e pode chegar até mais de R\$ 8.000 nos modelos mais sofisticados.

\textbf{3. Plataformas exclusivas} - Nesse caso, não há como definir uma media de preço, visto que o orçamento depende de inúmeros fatores. O que pode-se adiantar, é que o investimento é muito alto e por isso pouco aconselhável para quem está iniciando agora. Todos os custo que vão depender muito do tamanho do negócio que você pretende-se montar e dos recursos que serão necessário \cite{Guiadeecommerce2014}.

\subsection{Custos do Provedor de Hospedagem}

O provedor de hospedagem é um serviço oferecido para o armazenamento da loja virtual (Ou Web sites) na Internet, no qual deve garantir disponibilidade, segurança, rapidez de acesso, entre outras serviços mais (dependendo do plano escolhido). Um serviço de provedor e hospedagem deve ter um alto nível de disponibilidade, ou seja, ficar disponível pra acesso 24 horas por dia, pois um domínio fora do ar, causará prejuízo nos negócios. Na contratação de um serviço de provedor de hospedagem, vale a pena gastar um pouco mais, se isso for garantir um serviço de qualidade \cite{Revistapegn2014}.

Com base no ranking dos melhores provedores de hospedagem publicado por Barbosa (2014)\nocite{Barbosa2014}, o custo médio dos planos anuais oferecidos por estes serviços estão entre 9,90 à x + o registro do domínio.

\subsection{Custos Pagamento Digital}

Meios de pagamento ágeis e seguros fazem toda a diferença no comércio eletrônico. Quanto mais formas você oferecer, mais clientes vai satisfazer. O meio mais usado para o pagamento de compras virtuais atualmente é o cartão de crédito, mas não se deve descartar outras possibilidades, como os cartões de débito, de lojas, boletos, transferências bancárias e até o celular. 

Segundo Cruz, coordenador de \textit{marketing} na B2Log (2014)\nocite{Cruz2014}, existe três maneiras diferentes de se receber pelos produtos ou serviços oferecidos em sua loja virtual. São eles:

\textbf{1. Intermediadores de pagamento} - Solução na qual se terceiriza o serviço de pagamento do produto. Ou seja, O cliente é direcionado para o site do intermediador na etapa de efetuação do pagamento.

\textbf{2. \textit{Gateways} de pagamento} - Esta solução consiste em realizar o processo de pagamento digital de uma forma mais simplificada, são como as máquinas de cartão utilizadas no varejo online. Na etapa de efetuação do pagamento, o cliente insere as informações o cartão de crédito no próprio e-commerce, não sendo redirecionado pra nenhum sistema intermediador. O \textit{gateway} se comunica com a rede de adquirência (Rede ou Cielo, por exemplo) e verifica se existe saldo suficiente para realizar essa compra, e a processo de compra é finalizado.

\textbf{3. Integração direta com o adquirente, como Cielo e Rede} - Nesse caso, não há nenhum intermediador. Quando se opta em fazer a ligação direta com a adquirente, você só terá a taxa administrativa das adquirentes (Cielo e Rede, por exemplo). O pagamento será realizado direto pela loja virtual, que será capaz verificar se o cartão em questão tem saldo suficiente para a compra.

\subsection{Custos Logísticos}

A logística é um componente essencial do comércio eletrônico. O processo logísticos é basicamente composto pelas seguintes etapas:

\begin{enumerate}
	\item Recepção e condicionamento de produtos;
	\item Estocagem;
	\item \textit{Picking} (deslocamento de  produtos para a preparação do pedido);
	\item Intervenção das transportadoras assumindo a entrega.
	\item Recepção e condicionamento de produtos;
	\item Estocagem;
	\item \textit{Picking} (deslocamento de  produtos para a preparação do pedido);
	\item Intervenção das transportadoras assumindo a entrega.
\end{enumerate}

Essas etapas devem ser inseridas em ferramentas de rastreamento dos pedidos, que deverá estar integrada a plataforma de e-commerce escolhida, permitindo um melhor controle das diferentes operações, possibilitando ao cliente acompanhar em tempo real as fase em que se encontra o produto adquirido. Este componente do comércio eletrônico nem sempre é necessário, pois depende muito do segmento definido para o negócio. Visto que um segmento para a prestação de serviços \textit{online} ou comercialização de \textit{softwares}, não há uma entrega física do serviço ou produto (\textit{software}) ao cliente, e sim uma entrega "\textit{online}", como por exemplo: a comercialização de licença para uso de um determinado \textit{software}, ou a prestação de serviços de maneira \textit{online}. Portando, nesse caso não necessita de logística para a entrega dos mesmos. 

O lojista deve garantir que os produtos vendidos em seu comércio eletrônico cheguem ao cliente no prazo certo e em perfeitas condições. O custo disso não é baixo, e deve constar nas despesas totais de implementação da loja virtual. O serviço completo de logística inclui a recepção das mercadorias, a estocagem, o deslocamento dos produtos para preparação do pedido e, finalmente, a estrutura de transporte para a entrega dos itens.

O lojista pode optar por gerenciar sua própria logística ou delegar o serviço logístico à empresas terceiras. No início as atividades de uma loja virtual,normalmente a demanda de pedidos é baixa, ainda possibilitando o gerenciamento logístico internamento.  À medida que o negócio vai se tornando mais popular, aumentando o fluxo de vendas, a loja virtual passa a entregar um volume cada vez maior de pedidos. Nesses casos, a delegação do serviço logístico passa a ser uma opção interessante \cite{Guiadeecommerce2015}.

Segundo Coelho (2010), "O custo do estoque é composto por diversos elementos: (1) o próprio valor do estoque que poderia estar investido rendendo juros e pela oportunidade do capital; (2) manter o estoque também custa dinheiro: seguros, obsolescência, perdas e outros riscos associados; (3) durante a operação de transportes, um pouco do estoque fica indisponível dentro dos caminhões – assim, o estoque em trânsito também compõe este custo; (4) finalmente, caso os estoques não sejam bem gerenciados, a empresa terá uma falta de produtos, e este custo é difícil de ser mensurado." \nocite{Coelho2010}.


\subsection{Custos operacionais}

Fonte: http://guiadeecommerce.com.br/quanto-custa-montar-uma-loja-virtual/
\cite{Guiadeecommerce2014}

[Precisa Terminar]...

